\documentclass[a4paper, 12pt, titlepage]{article}

\usepackage[
    a4paper,
    lmargin=25.4mm,
    rmargin=25.4mm,
    tmargin=20mm,
    bmargin=20mm
]{geometry}

\usepackage[ddmmyyyy]{datetime}
\usepackage{caption}
\usepackage{color}
\usepackage{enumitem}
\usepackage{fancyhdr}
\usepackage{float}
\usepackage{graphicx}
\usepackage{listings}
\usepackage{listing}
\usepackage{multicol}
\usepackage{nameref}
\usepackage{parskip}
\usepackage{pgffor}
\usepackage{tocloft}

\IfFileExists{inconsolata.sty}{\usepackage{inconsolata}}

\newcommand{\code}[1]{\small\texttt{#1}\normalsize}

\definecolor{codegray}{gray}{0.9}
\definecolor{dkgreen}{rgb}{0,0.6,0}
\definecolor{gray}{rgb}{0.5,0.5,0.5}
\definecolor{mauve}{rgb}{0.58,0,0.82}

\lstset{inputpath={{../src/mp_ocr/}}}
\lstdefinestyle{numbers} {numbers=left, numberstyle=\ttfamily}
\lstdefinestyle{color}
{
    commentstyle=\color{dkgreen},
    keywordstyle=\color{blue},
    stringstyle=\color{mauve},
}

\lstdefinestyle{common}
{
    breakatwhitespace=false,
    breaklines=true,
    columns=fixed,
    showstringspaces=false,
    xleftmargin=0.65cm,
    basicstyle=\footnotesize\ttfamily,
    tabsize=4,
    postbreak=\mbox{\textcolor{gray}{$\hookrightarrow$}\space},
    literate={*}{*\allowbreak}1,
    numbersep=10pt,
}

\lstdefinestyle{code} {style=common, style=color, style=numbers}
\lstdefinestyle{raw} {style=common, style=color}

\fancyhf{}
\setlength\columnseprule{0.4pt}
\setlength{\parindent}{0pt}

\graphicspath{{../src/}}

\captionsetup{
    width=.8\linewidth,
    justification=centering
}

\title{\huge \textbf{Machine Perception Assignment
Digits Extraction And Recognition}}
\author{Julian Heng (19473701)}
\date{\today}

\begin{document}

\maketitle
\tableofcontents
\newpage

\pagestyle{fancy}

\fancyhf[HL]{\footnotesize{Machine Perception Assignment - Digits Extraction And Recognition}}
\fancyhf[FC]{\thepage}
\fancyhf[FL]{\footnotesize{Julian Heng (19473701)}}

\section{Discussion}
\subsection{Attempts}
I've started this assignment after completing Sam's OCR challenge, where digit
recognition was performed with contours and KNN. I've decided to reuse certain
components from that practical into this assignment. I've only used contours
since starting this assignment and did not experiment with other methods of
extracting digits using Harris corners or SIFT because I thought that using
contours provides very good results.

Most of the time spent on the assignment was extracting the digits from the
images. The hardest challenge when extracting the digits was trying to isolate
only the contours that contains the numbers without any contours containing
noise. I've tried filtering by contour features and have managed to just filter
only just the relevant contours, but it brought along the issue that the
filters were too specific to the provided images and will fail on other images.
Therefore, further ideas needs to be explored and will be discussed in the
section \nameref{impl}.

I've also used KNN to detect the digits because I've already completed the
aforementioned practical where KNN was already fully implemented.


\subsection{Implementation Details}
\label{impl}

There are a total 6 steps to extract digits in my implementation:
\begin{enumerate}
    \item Process the image using a Gaussian filter and Otsu binary
        threshold
    \item Extract the contours for the digits
    \item Crop the image to the contours
    \item Process the cropped image using a Gaussian filter and Otsu binary
        threshold
    \item Perform connected components on the crop
    \item Detect the number on each connected component
\end{enumerate}


\subsubsection{Image Preprocessing}
I've decided to use a Gaussian filter to remove noise from the images. I've
chosen 5 as my kernel size as any kernel size large than 7 will cause some of
the digits in the provided images to distort after binary threshold is applied.

\begin{figure}[H]
    \centering
    \fbox{\includegraphics[]{{DEBUG_bin_tr06}.jpg}}
    \caption{An example image after image preprocessing}
\end{figure}


\subsubsection{Contours Extraction}

\begin{figure}[H]
    \centering
    \fbox{\includegraphics[]{{DEBUG_contours_tr06}.jpg}}
    \caption{The contour groups extracted from the image}
\end{figure}

\begin{figure}[H]
    \centering{
        \fbox{\includegraphics[]{{DEBUG_cropped_0_tr06}.jpg}}
        \fbox{\includegraphics[]{{DEBUG_cropped_1_tr06}.jpg}}
    }
    \caption{The individually cropped contour groups from the image}
\end{figure}


\subsubsection{Digit Recognition}

\begin{figure}[H]
    \centering{
        \fbox{\includegraphics[]{{DEBUG_component_0_tr06}.jpg}}
        \fbox{\includegraphics[]{{DEBUG_component_1_tr06}.jpg}}
    }
    \caption{The connected components of the digits from the cropped image used to predict}
\end{figure}


\subsection{Results}


\newpage
\section{Source Code}

\subsection{colors.py}
\fancyhead[HR]{\footnotesize{colors.py}}
\lstinputlisting[language=Python,style=code]{colors.py}
\newpage


\subsection{image.py}
\fancyhead[HR]{\footnotesize{image.py}}
\lstinputlisting[language=Python,style=code]{image.py}
\newpage


\subsection{\_\_init\_\_.py}
\fancyhead[HR]{\footnotesize{\_\_init\_\_.py}}
\lstinputlisting[language=Python,style=code]{__init__.py}
\newpage


\subsection{\_\_main\_\_.py}
\fancyhead[HR]{\footnotesize{\_\_main\_\_.py}}
\lstinputlisting[language=Python,style=code]{__main__.py}
\newpage


\subsection{mp\_ocr.py}
\fancyhead[HR]{\footnotesize{mp\_ocr.py}}
\lstinputlisting[language=Python,style=code]{mp_ocr.py}
\newpage


\subsection{ocr.py}
\fancyhead[HR]{\footnotesize{ocr.py}}
\lstinputlisting[language=Python,style=code]{ocr.py}
\newpage


\subsection{train/\_\_init\_\_.py}
\fancyhead[HR]{\footnotesize{train/\_\_init\_\_.py}}
\lstinputlisting[language=Python,style=code]{train/__init__.py}
\newpage


\subsection{train/knn.py}
\fancyhead[HR]{\footnotesize{train/knn.py}}
\lstinputlisting[language=Python,style=code]{train/knn.py}
\newpage


\subsection{utils.py}
\fancyhead[HR]{\footnotesize{utils.py}}
\lstinputlisting[language=Python,style=code]{utils.py}
\newpage

\end{document}
